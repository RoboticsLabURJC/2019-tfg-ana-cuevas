\chapter*{Resumen}

Con este trabajo se pretende crear una nueva versión de las prácticas de la asignatura de Tratamiento Digital de la Imagen impartida en la URJC utilizando los cuadernillos de Jupyter con el objetivo de hacerlas más accesibles. La versión original de estos ejercicios se realiza en Matlab que al ser un programa que requiere una licencia de pago limita quién puede acceder a estas prácticas.\\

El objetivo principal de este trabajo es crear una forma gratuita de acceder a una base de conocimiento sobre el tratamiento de la imagen. Para ello se han planteado las prácticas en la plataforma de los cuadernillos de Jupyter que usan el lenguaje de programación Python, el lenguaje más usado en este momento y más accesible para principiantes. Otra ventaja de estos cuadernillos es que funcionan sobre un navegador por lo que se puede acceder a ellos desde cualquier sistema operativo.\\

El trabajo consiste de 9 prácticas, 7 de ellas son de imagen y 2 de video. Se ha intentado en todo momento ser fiel a las originales manteniendo lo que se pretendía enseñar con cada una de ellas.\\

El desarrollo de este TFG ha requerido familiarizarse con el entorno de Jupyter y con varias librerías de Python como Numpy y OpenCV además del desarrollo de nuevas funciones que existen en Matlab pero no se ha podido encontrar en Python.\\

Por último estas nuevas prácticas cuentan con unos enunciados extendidos para incluir más teoría y en inglés para ampliar el número de personas que puedan acceder a ellas.\\
